\chapter{Instrukcja wdro�eniowa}

W tej instrukcji zostanie pokazana procedura uruchomienia systemu w trybie lokalnym.

Do uruchomienia systemu s� wymagane nast�puj�ce narz�dzia:
\begin{itemize}
	\item \texttt{Node.js} (wersja v18.0.0) oraz \texttt{npm} (wersja 8.6.0) aby uruchomi� frontend,
	\item \texttt{Maven} (wersja 3.6.3) oraz \texttt{JDK} (minimum wersja 17) aby uruchomi� backend,
	\item \texttt{PostgreSQL} - postawiona baza danych o nazwie \texttt{programming\_technologies}, loginie \texttt{programming\_technologies} oraz ha�le \texttt{postgres},
	\item \texttt{MongoDB} - postawiona baza danych o nazwie \texttt{technologie-it}, loginie \texttt{programming\_technologies} oraz ha�le \texttt{postgres}.
\end{itemize}

W folderze \texttt{aplikacja} znajduj� si� nast�puj�ce foldery:
\begin{itemize}
	\item \texttt{frontend} - przechowuj�cy kod aplikacji klienckiej,
	\item \texttt{backend} - zawieraj�cy kod aplikacji serwerowej,
	\item \texttt{keycloak} - zawieraj�cy serwer Keycloak.
\end{itemize}

Aby uruchomi� aplikacj� klienck�, nale�y z folderu \texttt{frontend} uruchomi� w konsoli komendy \texttt{npm install} do zainstalowania zale�no�ci oraz 
nast�pnie~\texttt{npm start} do uruchomienia aplikacji.

W celu uruchomienia aplikacji serwerow�, nale�y z folderu \texttt{backend} uruchomi� w konsoli komend� \texttt{mvn spring-boot:run}.

Do uruchomienia serwera Keycloak, potrzebne b�dzie wywo�anie ze �cie�ki \texttt{keycloak/bin} komendy \texttt{kc.bat start-dev}.

Po tych czynno�ciach aplikacja b�dzie dost�pna pod adresem \texttt{localhost:3000}.